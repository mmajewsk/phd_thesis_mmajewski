\begin{savequote}[75mm]
In the beginning there was nothing, which exploded.
\qauthor{Terry Pratchett}
\end{savequote}

\chapter{Introduction}
\label{introduction}

In a brief moment, a universe happened.
After the Big Bang, the matter (leptons and quarks that form protons and neutrons) came to life following a brief period known as inflation.
The currently known physics says that an equal amount of antimatter (to compliment the matter) should be produced as well.
Yet, we do not observe the universe to be symmetrical, cosmic yin-yang; equal parts matter and antimatter.
This is one of the greatest riddles of current physics.
One of the goals of the LHCb experiment at CERN was to find and study the processes that can contribute to this asymmetry.

The grand promise of the modern physics is the theory of everything; a universe boiled down to a single theory, perhaps even to a single equation.
This dream is partially realised by the Standard Model, the theory that describes three fundamental forces (interactions) in physics.
Weak nuclear, strong nuclear, and electromagnetic force, come together into a single equation.
To some, it may be surprising that all of the mechanics in the known universe can be rooted just in a single theory based on the principle of local gauge invariance.
But the emergence of complex mechanisms from simple rules is prevalent in science.
Game of Life - a cellular automata simulation, although containing a small set of rules can make increasingly complex structures.
Most of the known chemistry that guides the biological processes, can be explained almost exclusively "just" using the electrical potential and probability.

As particle physicists, we see this emergence of complexity even at a very basic level.
By colliding two protons, we see a multitude of emergent possibilities, many different particles, and a vast sea of probable combinations of fundamental building blocks.
Studying those combinations of possibilities is the bread and butter of modern particle physics.
One of the misconceptions about High Energy Physics is that it is like a grandiose hunt through a murky and empty Forrest for one big prey.

In reality, it is more like being an explorer of new land.
Every day we check new jungle parts, expecting to find the trees and maybe some new species of the forest flora, and we see precisely that.
On some days, we revisit some parts to see if we notice anything extraordinary.
But still, we hope that we will find an ancient city of unknown civilisation with unimaginable treasure that will change everything that we know.
And we can't deny it does not exist until we search through all of the continent.
For particle physicists, this treasure is physics beyond the standard model.

This steady, monotone and patient work of checking all of the branches of physics is facilitated by the international community of scientists, of which CERN scientists constitute by far the largest group, with the biggest particle accelerator - LHC (Large Hadron Collider).
The LHC is a circular accelerator located in the underground tunnel under the Swiss-French border.
A large portion of the physicist working at CERN are parts of four main experiments Alice, Atlas, CMS and LHCb.
During my PhD studies, I was privileged to be one of the collaborators in the LHCb experiment.
The LHCb experiment discovered so far 54 new particles while analysing the collected data samples.
Such extraordinary science done with the LHCb spectrometer couldn't be possible without years of preparation and without overcoming technical challenges posed by the physics and the hardware.
Cooling systems, electronics, mechanic design, software, tracking, particle identification, all of the subsystems, and parts of the detector spawn new problems to be solved by the collaboration and the greater scientific community.
I am glad to introduce the reader to my thesis, describing some solutions to those problems.
