\begin{savequote}[75mm]
In the beginning there was nothing, which exploded.
\qauthor{Terry Pratchett}
\end{savequote}

\chapter{Introduction}
\label{introduction}

In a brief moment, a universe happened. After the bing bang, the matter (leptons that form protons and neutrons) came to life in a time of about 2-3  human breaths. The currently known physics says that while this is certainly possible, an equal ammount of antimatter should be produced as well. Yet, we do not observe the universe to be symmetrical, cosmic ying-yan; equal parts matter, and anti matter. This is one of the greatest riddles of current physics. One of the goals of the LHCb experiment at CERN was to find and study the processes that can contribute to this assymetry. 
The grand promise of modern physics is the theory of everything; a universe boiled down to single theory, pheraps even to a single equation. This dream is partially realised by the Standard Model, the theory that describes three fundamental forces in physics. Weak nuclear, strong uclear, and electromagnetic force, come together into single equation. To some, it may be suprising that all of the mechanics in the know universe can be rooted just in a small theory. But the emergence of complex mechanism from simple rules is prevailant in science. Game of Life - a cellural automata simulation, althoug containg small set of rules can make increasingly complex structures. Most of the known chemistry that guides the biological processes, can be explained almost exclusevily "just" using the electrical potential and probability.
As particle physicists, we see this emergence of complexity even at very basic level. Just by colliding two protons, we see a multitude of emergent possibilities, many different particles, and vast sea of probable combinations of fundemantal building blocks. Studying those combinations of possibilities is a bread and butter of modern particle physics. One of the misconceptions about High Energy Physics, is that it is like grandiose hunt through a murky and empty forrest for one big prey. In reality it is more like egg farming. Everyday we check new chicken coops, expeting eggs, and usually we find exactly that. On some days, we revisit old coops to be double sure that hens are still producing the eggs. But still, we hope, that there is this one chicken that makes gold eggs. And we can't deny it does not exist, until we check all of the coops.

This steady, monotone and patient work, of checking all of the branches of physics is facilitated by international community of scientist, of which CERN scientists constitute by far the largest group, with the biggest particle accelerator - LHC (Large Hadron Collider). The LHC is a circullar accelerator, located in the underground tunnel under Swiss-French border. The large portion of the physicist working at CERN are parts of four main experiments Alice, Atlas, CMS and LHCb. During my PhD studies I had a privilige to be one of the collaborators of the LHCb experiment. Such extraordinary science done with the LHCb spectrometer, couldn't be possible without years of preparation, and without overcoming technical challenges posed by the detectors hardware. Cooling system, electronics, mechanic design, all of the subsystem, and parts of the detector spawn new problems to be solved by the collaboration. I am glad to introduce the reader into my thesis, describing some of the solutions to those problems.  