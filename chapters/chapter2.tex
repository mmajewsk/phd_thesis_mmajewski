\begin{savequote}[75mm]
This is some random quote to start off the chapter.
\qauthor{Firstname lastname}
\end{savequote}

\chapter{The Large Hadron Collider beauty experiment}

\section{VELO}

\subsection{Velopix}

\textbf{THIS IS COPY PASTE WIP}
\subsubsection{Velo module}

The Large Hadron Collider and it's detectors were planned to be incrementally
improved over time. In Run III of the LHC, there will be a significant increase
in the luminosity, and the VELO must addapt to that change. For this purpose the
VeloPix ASIC was adapted from TimePix.


%% pic velo new modules
The new Velo for the Run III upgrade, has a modular design. Each module contains
12 VeloPix asics. VeloPix asics are rectangular silicon pixel detectors, 255x255 pixels,
each pixel has rectangular shape with $55 x 55 \mu m^{2}$ size. There is 52
modules, placed simillarilly as strip velo's module.

\subsubsection{Signal, trims and clustering}

The most drasti change in VeloPix operation is the goal of changing entirely the
readout type from analog to digital. VeloPix has several modes of readout, but
in the end the asynchronic hit position information will be the only signal coming from
the readout electronics. But this requires proper internal calibration of the
pixels.

\paragraph{VeloPix calibration}
\textbf{THIS SUBSECTION IS WIP}
Simillarilly to strip Velo, VeloPix will listen for the noise distribution. In
case of the VeloPix, each of the pixels has its own voltage threshold. If this
threshold is exceeded in a pixel, we register a hit.
There are two types of values that can be set in VeloPix to achievie optimal
threshold: Global Threshold and Trim.
Trims are indivudal values per each pixel. Trims are 4-bit (it has range of
16 values).
Global Threshold $GT$ is a voltage value that common for all of the pixels.
Independantly of the $GT$ value, all of the pixels can differ in
sensitivity.
First we set all pixels to use their $Trim_{0}$ value, and we scan the reaction of
the pixels by moderating $GT$ value, and holding it for constant
ammount of time $T_{C}$. Then we repeat this operation for $Trim_{15}$.
In both trim cases we record total number of active pixels (pixels that registered hit)
$N_{HITS}(GT)$ as a function of $GT$, as well as the number of registered hits ${H_P}_{x,y}(GT)$ in each
pixel $P_{x,y}$, in a $GT$.

The mean of the gaussian distribution of hits $H_{P_{x,y}}$ for each of the steps of $GT$ is
used to calculate lower bound for trim ${TrimValue_0}_{x,y}$ in case of
$Trim_{0}$, as well ass to calculate ${TrimValue_{15}}_{x,y}$ in $Trim_{15}$.
That means that scan in TRIM0 and TRIM15 is used to set the range of voltage for
the trims.
The ${TrimValue_{0}}_{x,y}$ is used as the lower range for all of the 16 trims in
each individual pixel and the ${TrimValue_{0}}_{x,y}$ is used as the maximal range
of voltage for the trims, and the step of the strim is
$TrimStep_{x,y} ={TrimValue_{15}}_{x,y}-{TrimValue_{0}}_{x,y}/15 $.

That process allows us to calculate calibrate trim values for each of the pixel.
Now in order to determine which of the trim value should be set for pixel, we
use the $N_{HITS}(GT)$ in each of the two scans ($Trim_{0}$ and $Trim_{15}$). Each of those scans
produces a gaussian distribution.
We calculate mean of the distribution in TRIM0 $M_{TRIM0}$, and TRIM15
$M_{TRIM15}$. The goal value of the pixels trims is calculated as
$GT_{target} = M_{TRIM0}+M_{TRIM15}/2$.

Then each of the pixels has its trim value ${PixelTrim_{x,y}}_{a}$ set as close to
$GT_{target}$.

Then in order to exclude the noise from the signal coming from the pixels, the
effective global threshold $GT_{effective}$ is set to
$GT_{target} + 1000 electrons$.
So the actual threshold for the individual pixel
$Th_{x,y} = {PixelTrim_{x,y}}_{a} + GT_{target} + 1000 electrons$
)
\paragraph{Calibration example}
Let's say that we start the scan with global threshold value of $GT = 1100 DAC$, and increase to $1600 DAC$, with a
step of $1 DAC$.

%% @TODO ref plot.
%% @TODO finish example
For example, if the mean
from TRIM0 was 1111, and mean from TRIM15 was 1431, this means that the trim
step is (1411-1111/15) == 20. This means that the TRIM1 is 1131, TRIM2 is 1151 and etc.
\subsection{Monitoring and calibration problems}



\section{RICH}
\section{Trackers}
\section{Calorimeters}