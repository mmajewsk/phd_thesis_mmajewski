% the acknowledgments section

\noindent \textit{I acknowledge support from National Science Centre (Poland), UMO-2017/27/N/ST2/01880.}\\
\noindent \textit{This research was supported in part by ACK CYFRONET AGH (PLGrid Infrastructure).}
\\
\\

\newthought{One of the key insights from one of my favourite books is that}, no matter how small your stride is, as long as you take enough steps, you’ll be able to travel further than you ever imagined.
Saying this, I would like to express my gratitude to everyone who has helped, accompanied or made any of the steps towards finishing this work more enjoyable.
\bigbreak
First, I would like to thank my supervisor Tomasz, without whom this journey wouldn’t be possible. Thank you for the countless times you have helped me, offered me guidance, believed in my abilities, and gave me space for growth.

I would also like to thank my friend from room 121. Thank you, Bartek, for helping, especially with the VeloPix section of this work, and for often being someone I confide in my problems with.

This journey would undoubtedly be significantly less enjoyable and more difficult if it weren’t for Wojtek, Paweł and other members of the LHCb-AGH and KOIDC. Thank you for being so helpful with answering many of my questions and with formal issues, and also for being excellent travel companions. I wish you good luck in your future careers.

The majority of the scope of this work touches on the LHCb experiment at CERN, so the thanks are in place for the entirety of the LHCb collaboration, especially to Victor and Kazu.
Being part of this collaboration has been the highlight of my scientific career. It always will be an experience that I cherish and am proud of.

Additionally, I would like to thank all of the students that partially contributed to this work (in no particular order): Jakub, Dominik, Mateusz, Julian, Patryk.

\bigbreak
Now I wish to thank all the people that influenced the years of my PhD studies on a more personal level.

I want to thank my friends from CSC: Andrzej, Kimmo, Oliver, and Willem. Thank you for making my experience with the CERN community so much unique.
Big thanks to Artur and all of the members of the Pykonik and Pydata Kraków community for being inspiriation for constant improvement of my CS skills.
Thanks to Jery for support in both good and challenging times.

The vast amounts of screen time spent on this work would be unbearable without the anchoring in reality, so I would like to thank all of the friends from Hawiarska Koliba for being that anchor. Especially thanks to those with which I could share my thoughts about the PhD experience (Kasia and Dr Czox).

Being a PhD student is a choice of way of life, and so I would like to thank my extended family for being supportive of this choice, and always rooting for me.

There is one person without whom I would have never even thought about picking up a PhD in physics. My highshcool teacher Mr. Andrzej. Thank you for showing me that the passion, enthusiasm and sheer joy of the work are far more critical than having top grades.

I would also like to thank my first teachers: mom and dad. Thank you, dad, for sparking my interest in science and technology, and thank you, mom, for equiping me with ambition and willpower to make great use of it.

Thank you, sister, for often giving me a quiet place to study, and thank you for always having my back, even when on the other side of the continent.

And last but by far not least, thank you, Ola, for your constant, unyielding,
unrelenting, rock-solid support throughout the entire course of my PhD, and especially during the last months.

